\documentclass{article}

\usepackage[utf8]{inputenc}
\usepackage{geometry}
\geometry{a4paper, margin=1in}
\usepackage{fancyhdr}
\pagestyle{fancy}

\title{Snapshots (1/2/3/4)}
\author{Group 5: Gabe Foreman, Lester Low, Jose Oscanoa, Christine Vu}
\date{November 21, 2024}

\begin{document}
\fancyhead[R]{CS 3338 Snapshots}
\maketitle
\tableofcontents
\newpage


\section{Version Description}

\begin{table}[h!]
    \centering
    \begin{Large}
    \begin{tabular}{|c|c|c|}
    \hline
    
    Snapshot & Description & Date Added \\ \hline
     1 & Start Objectives & \date{August 21, 2024} \\ \hline
     2 & 1st Checkpoint & \date{October 21, 2024} \\ \hline
     3 & 2nd Checkpoint & \date{November 21, 2024} \\ \hline
      4 & Due Date Checkpoint & \date{December 11, 2024} \\ \hline
    
    \end{tabular}
    \end{Large}
    \label{tab:my_label}
\end{table}

\newpage


\section{Introduction & Product Description}
The arQive is an online digital storytelling map dedicated to LGBTQ+ narratives,
offering a collection of queer stories that are geo-located and digitally preserved. Dr.
Cynthia Wang established this platform in 2014 and was later joined by co-founder
Zachary Vernon in 2019. The platform serves as a space where individuals can
share personal, historical, and community stories while accessing information about
safe spaces. It functions as a digital archive of past and present movements,
personal experiences, resources, and organizations from around the globe. The
arqive aims to both create and collect these stories and resources, serving as a
testament to the importance of the LGBTQ+ community.

The arqive's primary mission is to empower LGBTQ+ individuals to map out and
share their stories, providing them with a meaningful place in both the world and
history. The archive is available as both a web and mobile application. This marks
the fourth year of the arqive's evolution as a senior design project. In 2020, the
project debuted with the launch of the website. The following year, in 2021, the
development team introduced a mobile application for both Android and iOS
platforms to complement the website. This year, our team focused on enhancing
both the website and mobile app by incorporating additional features and addressing
various bugs for an improved user experience.



\section{Snapshot 1}

\subsection{Goals \& Objectives}
\indent

\textbf{Develop Core Application Features:}
    \begin{itemize}
        \item Design a user-friendly interface for both web and mobile devices.
        \item Allow users to post stories, content, and events on the website. This specifically includes images, gifs, videos, and audio.
        \item Implement a map-based feature to mark locations and experiences by users.
        \item Ensure scalability for any user base \end{itemize}

\section{Snapshot 2}

\subsection{Completed Goals}
    \begin{itemize}
        \item Completed Basic landing page (web based only)
        \item Basic MAP integration via Google API
        \item Plaintext stories from user input, users cannot mark locations
    \end{itemize}
    
\subsection{Current Objectives}
\textbf{New Goals:}
    \begin{itemize}
        \item Create a separate NSFW tag for users to include (for app store approval)
        \item Develop functionality for mobile based apps
        \item Redesign UI/UX to a minimal aesthetic
    \end{itemize}
\textbf{In Progress Objectives:}
    \begin{itemize}
        \item Allow for users to post images, gifs, videos and audio.
        \item Allow for users to mark locations for given stories/input
    \end{itemize}
    
\section{Snapshot 3}
\subsection{Completed Goals}
    \begin{itemize}
        \item Completed Basic landing page 
        \item Basic MAP integration via Google API
        \item Stories from user input. Only includes plaintext 
        \item Users can mark locations for given stories/input
        \item Developed functionality for mobile based apps
    \end{itemize}
    
\subsection{Current Objectives}
\textbf{New Goals:}
    \begin{itemize}
        \item Re-implement map using Google Map Tiles API+
        \end{itemize}
\textbf{In Progress Objectives:}
    \begin{itemize}
        \item Create a separate NSFW tag for users to include (for app store approval)
        \item Redesign UI/UX to a minimal aesthetic
        \item Allow users to upload multimedia stories (images, gifs, videos, etc.)
    \end{itemize}

\section{Snapshot 4 - Final Snapshot}
\subsection{Completed Goals}
    \begin{itemize}
        \item Completed Basic landing page 
        \item Basic MAP integration via Google API
        \item Stories from user input. Only includes plaintext 
        \item Users can mark locations for given stories/input
        \item Developed functionality for mobile based apps
        \item Re-implement map using Google Map Tiles API
        \item Create a separate NSFW tag for users to include (for app store approval)
        \item Redesign UI/UX to a minimal aesthetic
    \end{itemize}
    
\subsection{Future Objectives}
    \begin{itemize}
        \item Allow users to upload multimedia stories (images, gifs, videos, etc.)
        \item Context Moderdation.
        \item Implementing more tags than just NSFW to curate user needs and trends.
        \item Improve Mobile Development
    \end{itemize}


\section*{Conclusion}
By the end of the semester most of the goals we set out to complete were
completed. Team 1 was able to completely build and deploy the nsfw function for the
website which will later serve as an outline for the nsfw function that will be
implemented into mobile. The back end for multimedia upload was completed using
last year's back end ground work which allows for front end code to begin. Team 2
was able to secure the authentication issue that the website faced over the summer.
Additionally, they were able to build the blocking function front end and back end.
Similar to the nsfw function this will be used as an outline for how it will be
implemented on mobile using react native. Small mobile functions were also
implemented by a team mate on team 2. There were many little things that had to be
modified/fixed along the way that were not accounted for in the list.

Although the main purpose of this class is to develop code for a company and fulfill
their wishes, the team was also able to gain valuable experience that can be taken
into their careers after graduation. Many were able to learn to react and react native
for the first time since it is not covered in our classes. Our build and deploy person
was able to learn how to use sandbox and digital ocean. Group members were able
to learn a lot about authentication and the trickiness of back end coding. Some
people became very efficient at doing both front end and back end for websites. In
addition team leads were able to learn skills like how to communicate with liaisons
and basic skills like how to take meeting minutes.

\end{document}